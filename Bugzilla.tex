% File: Bugzilla.tex
\chapter{Bugzilla Setup}
\label{ch:bugzilla}

This chapter details the configuration of the Bugzilla issue tracking system using Docker Desktop on a local machine.

\section{Docker Desktop Setup}

Bugzilla was deployed as a Docker container using the official \texttt{linuxserver/bugzilla} image. The container was started with the following command:

\begin{verbatim}
docker run -d --name bugzilla -p 8080:80 -e TZ=UTC linuxserver/bugzilla
\end{verbatim}

This command runs the Bugzilla container in detached mode, mapping port 8080 on the host to port 80 inside the container. The environment variable \texttt{TZ=UTC} sets the timezone.

\section{Accessing Bugzilla}

Once the container is running, Bugzilla can be accessed at:

\begin{center}
\url{http://YOUR_PUBLIC_IP:8080}
\end{center}

Replace \url{YOUR_PUBLIC_IP} with the actual public IP address of the machine hosting Docker Desktop. Make sure that port 8080 is open and forwarded properly if behind a router or firewall.

\section{Additional Notes}

\begin{itemize}
    \item Ensure Docker Desktop is running and the Bugzilla container is active.
    \item Verify that your machine's firewall or network configuration allows inbound traffic on port 8080.
    \item This setup provides a quick and isolated environment to use Bugzilla without a dedicated server.
\end{itemize}

