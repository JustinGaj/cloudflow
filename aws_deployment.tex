% File: aws_deployment.tex
\chapter{AWS Deployment}
\label{ch:aws-deployment}

This chapter covers the deployment of the ``Two Buttons'' website using AWS 
and a refactor of the JavaScript logic into a class-based design.

% --------------------------------------------------
\section{Deployment Steps}

\subsection{Dockerfile}
\begin{minted}{docker}
FROM node:20-alpine
WORKDIR /app
COPY package*.json ./
RUN npm install --omit=dev
COPY . .
EXPOSE 3000
CMD ["npm", "start"]
\end{minted}

\subsection{Build and Run Locally}
\begin{minted}{bash}
docker build -t color-buttons-app .
docker run -p 8080:3000 color-buttons-app
# visit http://localhost:8080
\end{minted}

\subsection{Setting Up AWS}
\begin{enumerate}
    \item Create an AWS account
    \item Create an SSO user in the IAM Identity Center, granting privileges as needed
    \item Build a container using the Elastic Container Registry
    \item Create a repository using the AWS App Runner and link it to the container
\end{enumerate}

\subsection{Push to ECR and Deploy with App Runner}
\begin{minted}{bash}
export AWS_REGION=us-east-1
export ECR_REPO=color-buttons-app
export IMAGE_TAG=v1
export AWS_ACCOUNT_ID=$(aws sts get-caller-identity --query Account --output text)

aws ecr create-repository --repository-name $ECR_REPO --region $AWS_REGION || true

aws ecr get-login-password --region $AWS_REGION \
| docker login --username AWS \
  --password-stdin $AWS_ACCOUNT_ID.dkr.ecr.$AWS_REGION.amazonaws.com

docker build -t $ECR_REPO:$IMAGE_TAG .
docker tag $ECR_REPO:$IMAGE_TAG \
  $AWS_ACCOUNT_ID.dkr.ecr.$AWS_REGION.amazonaws.com/$ECR_REPO:$IMAGE_TAG
docker push $AWS_ACCOUNT_ID.dkr.ecr.$AWS_REGION.amazonaws.com/$ECR_REPO:$IMAGE_TAG
\end{minted}

You can view our deployed website by clicking
\href{https://wrmtv5qpqi.us-east-1.awsapprunner.com/}{here}

Here is the website link as well:
https://wrmtv5qpqi.us-east-1.awsapprunner.com/

% --------------------------------------------------
\section{Two Buttons Refactor}

\subsection{UML Diagram}
\begin{table}[h!]
\centering  % centers the table on the page
\begin{tabular}{|c|}
\hline
TwoButtonApp \\ \hline
\begin{minipage}{6cm}  % adjust width as needed
\begin{itemize}
  \renewcommand{\labelitemi}{--}
  \item countA: number
  \item countB: number
  \item btnA: HTMLElement
  \item btnB: HTMLElement
\end{itemize}
\end{minipage} \\ \hline
\begin{minipage}{6cm}  % adjust width as needed
\begin{itemize}
  \renewcommand{\labelitemi}{+}
  \item constructor(buttonAId...)
  \item onButtonAClick(): void
  \item onButtonBClick(): void
\end{itemize}
\end{minipage} \\ \hline
\end{tabular}

\caption{UML Diagram of JS refactor}
\end{table}

\subsection{index.html}
\begin{minted}{html}
<!DOCTYPE html>
<html>
  <body>
    <h1>Two Buttons</h1>
    <button id="blueBtn">Blue</button>
    <button id="redBtn">Red</button>
    <div id="status"></div>
    <script src="app.js"></script>
    <script>
      new TwoButtonApp("blueBtn","redBtn");
    </script>
  </body>
</html>
\end{minted}

\subsection{app.js}
\begin{minted}{javascript}
class TwoButtonApp {
  constructor(buttonAId, buttonBId) {
    this.countA = 0;
    this.countB = 0;
    this.btnA = document.getElementById(buttonAId);
    this.btnB = document.getElementById(buttonBId);
    this.btnA.addEventListener("click", () => this.onButtonAClick());
    this.btnB.addEventListener("click", () => this.onButtonBClick());
  }
  onButtonAClick() {
    document.body.style.backgroundColor = "blue";
    document.getElementById("status").textContent =
      `Blue clicked ${++this.countA} times`;
  }
  onButtonBClick() {
    document.body.style.backgroundColor = "red";
    document.getElementById("status").textContent =
      `Red clicked ${++this.countB} times`;
  }
}
\end{minted}

